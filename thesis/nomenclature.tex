%
% Conventions
% ===========
%   . Definition order matters!
%   . Know the difference between \rm, \textrm, and \mathrm
%   . Use consistent notation
%   . Use \newcommand (\def is not allowed in RevTex)
%   . Use small space \, instead of big space \
%   . "vp" prefix denotes momentum vector
%
% Table of contents
% =================
% - Basic commands
% - Commands with arguments
%
% - Math
%   . Standard
%   . Units
%   . Stats
%
% - Variables
%   . Abbreviations
%   . MC generators
%   . Oscillation Parameters
%
% - Documentation
%   . Horizontal lines
%   . Margin comments
%   . Hyphenation
%

% -----------------------------------------------------------------------------
% Basic commands
\newcommand{\beq}{\begin{equation}}
\newcommand{\eeq}{\end{equation}}
\newcommand{\beqa}{\begin{eqnarray}}
\newcommand{\eeqa}{\end{eqnarray}}
\newcommand{\n}{\noindent}

% -----------------------------------------------------------------------------
% Math

\newcommand\varpm{\mathbin{\vcenter{\hbox{%
  \oalign{\hfil$\scriptstyle+$\hfil\cr
          \noalign{\kern-.3ex}
          $\scriptscriptstyle({-})$\cr}%
}}}}
\newcommand\varmp{\mathbin{\vcenter{\hbox{%
  \oalign{$\scriptstyle({+})$\cr
          \noalign{\kern-.3ex}
          \hfil$\scriptscriptstyle-$\hfil\cr}%
}}}}

% Units
\newcommand\unit[1]{\mbox{\,#1}}
\newcommand\evsq{\mbox{\,eV\textsuperscript{2}}}

% Stats
\newcommand\RMS          {r.m.s{.}}
\newcommand\CLs          {\mathrm{CL}_{\tiny S}}
\newcommand\likelihood   {\mathcal{L}}
\newcommand\Lmax         {\likelihood_\mathrm{max}}
\newcommand\BF           {\mathcal{B}}
\newcommand\Acc          {\mathcal{A}}
\newcommand\Cor          {\mathcal{C}}
\newcommand\Lint         {\mbox{\raisebox{2pt}{\footnotesize$\int$}} L\,{\rm{d}}t}
\newcommand\sigmu        {\upmu}
\newcommand\sigmuH       {\sigmu_{\scH}}
\newcommand\qmu          {q_\sigmu}

% Derivative is upright
\newcommand\D      {{\rm d}}

% -----------------------------------------------------------------------------
% Variables

% Abbreviation
\newcommand\mysymb{\scalebox{.3}{(}\raisebox{-1.7pt}{$-$}\scalebox{.3}{)}}

\newcommand\nova{NO$\nu$A}
\newcommand\NU[1]{\nu_{#1}}
\newcommand\ANU[1]{\bar{\nu}_{#1}}
\newcommand\nue{\nu_{e}}
\newcommand\numu{\nu_{\mu}}
\newcommand\nutau{\nu_{\tau}}
\newcommand\anue{\bar{\nu}_{e}}
\newcommand\anumu{\bar{\nu}_{\mu}}
\newcommand\anutau{\bar{\nu}_{\tau}}
\newcommand\nuanu{^{\text{\tiny{(}}}\bar{\nu}^{\text{\tiny{)}}}}
\newcommand\pot[1]{$#1 \times 10^{20}$ POT}

% Oscillation Parameters
\newcommand\dmxy[2]{\Delta m^2_{#2#1}}
\newcommand\thxy[2]{\theta_{#1#2}}
\newcommand\Uxy[2]{U_{#1#2}}
\newcommand\Usqxy[2]{\lvert \Uxy{#1}{#2} \rvert^2}
\newcommand\Ustxy[2]{\Uxy{#1}{#2}^{*}}

% MC generators
\newcommand\GEANT        {{\sc geant}}

% Known particle mass symbols

% Decay channels

% Regions
\newcommand\SR           {\textrm{\footnotesize\sc sr}}

% Super/subscripts

% Standard

% Special

% Fractions

% Efficiencies

% -----------------------------------------------------------------------------
% Colors

%\definecolor{myWW}{rgb}{0,0,255}

% -----------------------------------------------------------------------------
% Piechart

\newcommand{\slice}[6]{
  \pgfmathparse{0.5*#1+0.5*#2}
  \let\midangle\pgfmathresult

  % slice
  \ifthenelse{\equal{#2}{0}}{}{
    \draw[fill=#6] (0,0) -- (#1:1) arc (#1:#2:1) -- cycle;}

  % outer label
  \node[label=\midangle:#5] at (\midangle:0.95) {};

  % inner label
  \ifthenelse{#3 < 5}{}{
    \pgfmathparse{min((#2-#1-20)/110*(-0.3),0)}
    \let\temp\pgfmathresult
    \pgfmathparse{max(\temp,-0.4) + 0.8}
    \let\innerpos\pgfmathresult
      \node at (\midangle:\innerpos){
        \ifthenelse{\equal{blue}{#6}}{\color{white} #4}{
          \ifthenelse{\equal{magenta}{#6}}{\color{white} #4}{
            \ifthenelse{\equal{cyan}{#6}}{\color{white} #4}{
              \ifthenelse{\equal{black}{#6}}{\color{white} #4}{
                #4
              }
            }
          }
        }
      };
  }
}

% -----------------------------------------------------------------------------
% Documentation

% Clear two-column page
\makeatletter
\newcommand*{\balancecolsandclearpage}{%
  \close@column@grid
  \clearpage
  \twocolumngrid
}
\makeatother

% Repeating comments
\newcommand\HwwTableUpdate{\centering{\color{red}\large THIS TABLE WILL BE UPDATED}\\ }
\newcommand\HwwPlotUpdate{\centering{\color{red}\large FIGURE STYLE WILL BE UPDATED}\\ }
\newcommand\HwwPlotDetail[1]{#1 Fig.~\ref{fig:MET} for plotting details}
\newcommand\HwwPlotSelectionDetail[1]{#1 Figs.~\ref{fig:MET} and~\ref{fig:0j} for plotting details}
\newcommand\HwwPlotFitDetail[1]{#1 Figs.~\ref{fig:MET} and~\ref{fig:mT_fitDF} for details of plotting and normalizations}
\newcommand\HwwPlotFitDetailShort[1]{#1 Figs.~\ref{fig:MET} and~\ref{fig:mT_fitDF} for plotting details}
\newcommand\paper        {paper}
\newcommand\Paper        {Paper}

% Spacing
\newcommand\no           {\!\!}
\newcommand\np           {\no\no}
\newcommand\nq           {\np\np}
\newcommand\nqq          {\nq\nq\nq}
\newcommand\nqqq         {\nq\nq\nq\nq}
\newcommand\nr           {\phantom{i}\nq}
\newcommand\zz           {\phantom{..}}
\newcommand\z            {\phantom{0}}
\newcommand\Z            {\phantom{.0}}
\newcommand\QUAD         {~\,}

% Journals
\newcommand\Note[2]      {Report No.\ #2, \href{#1}{#1}}
\newcommand\CPC[1]       {Comput.~Phys.~Commun.~{\bf #1}}
\newcommand\EPJC[1]      {Eur.~Phys.~J.~C {\bf #1}}
\newcommand\JHEP[1]      {J.~High Energy Phys.~#1}
\newcommand\JINST[1]     {JINST {\bf #1}}
\newcommand\JPHYSG[1]    {J.~Phys.~{\bf G~#1}}
\newcommand\NIMA         {Nucl.~Instrum.~Methods~Phys.~Res.,~Sect.~A~}
\newcommand\NPhys[1]     {Nucl.~Phys.\ {\bf #1}}
\newcommand\PLB[1]       {Phys.~Lett.~B {\bf #1}}
\newcommand\PLett[1]     {Phys.~Lett.\ {\bf #1}}
\newcommand\PRep[1]      {Phys.~Rep.\ {\bf #1}}
\newcommand\PRev[1]      {Phys.~Rev.\ {\bf #1}}
\newcommand\PRD[1]       {Phys.~Rev.~D {\bf #1}}
\newcommand\PRL[1]       {Phys.~Rev.~Lett.\ {\bf #1}}
\newcommand\ZPC[1]       {Z.~Phys.~C {\bf #1}}
\newcommand\arxiv[1]     {\href{http://arxiv.org/abs/#1}{arXiv:#1}}
\newcommand\hepex[1]     {\href{http://arxiv.org/abs/hep-ex/#1}{hep-ex/#1}}
\newcommand\hepph[1]     {\href{http://arxiv.org/abs/hep-ph/#1}{hep-ph/#1}}

% Latin
\newcommand\ie           {i.\,e{.}}
\newcommand\eg           {e.\,g{.}}
\newcommand\etal         {{\it et al{.}}}
\newcommand\ibid         {{\it ibid{.}}}
\newcommand\insitu       {{\it in situ}}
\newcommand\cf           {cf{.}}

% Horizontal lines (double line, single line)
\newcommand\dbline{\noalign{\vskip 0.10truecm\hrule\vskip 0.05truecm\hrule\vskip 0.10truecm}}
\newcommand\sgline{\noalign{\vskip 0.10truecm\hrule\vskip 0.10truecm}}
\newcommand\clineskip{\noalign{\vskip 0.10truecm}}