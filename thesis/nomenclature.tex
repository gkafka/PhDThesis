%
% Conventions
% ===========
%   . Definition order matters!
%   . Know the difference between \rm, \textrm, and \mathrm
%   . Use consistent notation
%   . Use \newcommand (\def is not allowed in RevTex)
%   . Use small space \, instead of big space \
%   . "vp" prefix denotes momentum vector
%
% Table of contents
% =================
% - Basic commands
% - Commands with arguments
%
% - Math
%   . Standard
%   . Units
%   . Stats
%
% - Variables
%   . Abbreviations
%   . MC generators
%   . Oscillation Parameters
%
% - Documentation
%   . Journals
%   . Margin comments
%   . Hyphenation
%

% -----------------------------------------------------------------------------
% Basic commands
\newcommand{\beq}{\begin{equation}}
\newcommand{\eeq}{\end{equation}}
\newcommand{\beqa}{\begin{eqnarray}}
\newcommand{\eeqa}{\end{eqnarray}}
\newcommand{\n}{\noindent}

% -----------------------------------------------------------------------------
% Math

\newcommand\varpm{\mathbin{\vcenter{\hbox{%
  \oalign{\hfil$\scriptstyle+$\hfil\cr
          \noalign{\kern-.3ex}
          $\scriptscriptstyle({-})$\cr}%
}}}}
\newcommand\varmp{\mathbin{\vcenter{\hbox{%
  \oalign{$\scriptstyle({+})$\cr
          \noalign{\kern-.3ex}
          \hfil$\scriptscriptstyle-$\hfil\cr}%
}}}}

% Units
\newcommand\unit[1]{\mbox{\,#1}}
\newcommand\evsq{\mbox{\,eV\textsuperscript{2}}}
\newcommand\microsec{\,\mu\mbox{s}}

% Stats
\newcommand\RMS          {r.m.s{.}}
\newcommand\CLs          {\mathrm{CL}_{\tiny S}}
\newcommand\likelihood   {\mathcal{L}}
\newcommand\Lmax         {\likelihood_\mathrm{max}}
\newcommand\BF           {\mathcal{B}}
\newcommand\Acc          {\mathcal{A}}
\newcommand\Cor          {\mathcal{C}}
\newcommand\Lint         {\mbox{\raisebox{2pt}{\footnotesize$\int$}} L\,{\rm{d}}t}
\newcommand\sigmu        {\upmu}
\newcommand\sigmuH       {\sigmu_{\scH}}
\newcommand\qmu          {q_\sigmu}

% Derivative is upright
\newcommand\D      {{\rm d}}

% -----------------------------------------------------------------------------
% Variables

% Abbreviation
\newcommand\mysymb{\scalebox{.3}{(}\raisebox{-1.7pt}{$-$}\scalebox{.3}{)}}

\newcommand\nova{NO$\nu$A}
\newcommand\NU[1]{\nu_{#1}}
\newcommand\ANU[1]{\bar{\nu}_{#1}}
\newcommand\nue{\nu_{e}}
\newcommand\numu{\nu_{\mu}}
\newcommand\nutau{\nu_{\tau}}
\newcommand\anue{\bar{\nu}_{e}}
\newcommand\anumu{\bar{\nu}_{\mu}}
\newcommand\anutau{\bar{\nu}_{\tau}}
\newcommand\nuanu{^{\text{\tiny{(}}}\bar{\nu}^{\text{\tiny{)}}}}
\newcommand\pot[1]{$#1 \times 10^{20}$ POT}

% Oscillation Parameters
\newcommand\dmxy[2]{\Delta m^2_{#2#1}}
\newcommand\thxy[2]{\theta_{#1#2}}
\newcommand\Uxy[2]{U_{#1#2}}
\newcommand\Usqxy[2]{\lvert \Uxy{#1}{#2} \rvert^2}
\newcommand\Ustxy[2]{\Uxy{#1}{#2}^{*}}

% MC generators
\newcommand\GEANT        {{\sc geant}}

% Known particle mass symbols

% Decay channels

% Regions
\newcommand\SR           {\textrm{\footnotesize\sc sr}}

% Journals
\newcommand\CPC{Comput.~Phys.~Commun.}
\newcommand\JCyber{J.~Cybern.}
\newcommand\JHEP{J.~High Energy Phys.}
\newcommand\JINST{JINST}
\newcommand\NIMA{Nucl.~Instrum.~Methods~Phys.~Res.,~Sect.~A}
\newcommand\NPhys{Nucl.~Phys.}
\newcommand\NPhysB{Nucl.~Phys.~B}
\newcommand\PLett{Phys.~Lett.}
\newcommand\PLB{Phys.~Lett.~B}
\newcommand\PRep{Phys.~Rep.}
\newcommand\PRev{Phys.~Rev.}
\newcommand\PRD{Phys.~Rev.~D}
\newcommand\PRL{Phys.~Rev.~Lett.}
\newcommand\ZPC{Z.~Phys.~C}
\newcommand\arxiv[1]{\href{http://arxiv.org/abs/#1}{arXiv:#1}}
\newcommand\hepex[1]{\href{http://arxiv.org/abs/hep-ex/#1}{hep-ex/#1}}
\newcommand\hepph[1]{\href{http://arxiv.org/abs/hep-ph/#1}{hep-ph/#1}}
