%
% Conventions
% ===========
%   . Definition order matters!
%   . Know the difference between \rm, \textrm, and \mathrm
%   . Use consistent notation
%   . Use \newcommand (\def is not allowed in RevTex)
%   . Use small space \, instead of big space \
%   . "vp" prefix denotes momentum vector
%
% Table of contents
% =================
% - Basic commands
% - Commands with arguments
%
% - Math
%   . Standard
%   . Units
%   . Stats
%   . Bent arrow
%
% - Variables
%   . MC generators
%
%
% - Documentation
%   . Horizontal lines
%   . Margin comments
%   . Hyphenation
%

% -----------------------------------------------------------------------------
% Basic commands
\newcommand{\beq}{\begin{equation}}
\newcommand{\eeq}{\end{equation}}
\newcommand{\beqa}{\begin{eqnarray}}
\newcommand{\eeqa}{\end{eqnarray}}
\newcommand{\n}{\noindent}

% Commands with arguments
\newcommand\ABS  [1]{|\,{#1}\,|}            % absolute value
\newcommand\CEN  [1]{C_{#1}}                % centrality
\newcommand\MBF  [1]{\hbox{\bm{$#1$}}} % boldface lower-case Greek
\newcommand\SPACE[1]{\,{#1}\,}              % small space around
\newcommand\MBFr[1]{\multicolumn{1}{r}{\MBF{#1}}}
\newcommand\MCOL[2]{\multicolumn{#1}{l}{#2}}
\newcommand\MCOLc[2]{\multicolumn{#1}{c}{#2}}
\newcommand\MROW[2]{\multirow{#1}{*}{#2}}
%
\newcommand\mcolz{\multicolumn{2}{c}{~-}}

% -----------------------------------------------------------------------------
% Math

% Standard symbols
\newcommand\TO     {\SPACE{\rightarrow}}
\newcommand\MINUS  {\SPACE{-}}
\newcommand\PLUS   {\SPACE{+}}
\newcommand\OTIMES {\SPACE{\otimes}}
\newcommand\CDOT   {\SPACE{\cdot}}
\newcommand\CAP    {\SPACE{\cap}}
\newcommand\CUP    {\SPACE{\cup}}
\newcommand\PM     {\SPACE{\pm}}
\newcommand\EQUIV  {\SPACE{\equiv}}
\newcommand\EQ     {\SPACE{=}}
\newcommand\APPROX {\SPACE{\approx}}
\newcommand\GT     {\SPACE{>}}
\newcommand\LT     {\SPACE{<}}
\newcommand\GE     {\SPACE{\ge}}
\newcommand\LE     {\SPACE{\le}}
\newcommand\GG     {\SPACE{\gg}}
\newcommand\LL     {\SPACE{\mathchar"321C}}
\newcommand\GTRSIM {\SPACE{\gtrsim}}
\newcommand\TIMES  {\SPACE{\times}}
\newcommand\data   {\rm data}
%	\newcommand\RTO    {\,{\raisebox{7px}{\rotatebox{180}{$\Lsh$}}}\,} % angled right arrow

% Units
\newcommand\rad    {\mbox{\,rad}}
\newcommand\cm     {\mbox{\,cm}}
\newcommand\mm     {\mbox{\,mm}}
\newcommand\ns     {\mbox{\,ns}}
\newcommand\seconds{\mbox{\,s}}
\newcommand\mb     {\mbox{\,mb}}
\newcommand\pb     {\mbox{\,pb}}
\newcommand\fb     {\mbox{\,fb}}
\newcommand\iab    {\mbox{\,ab$^{-1}$}}
\newcommand\ifb    {\mbox{\,fb$^{-1}$}}
\newcommand\inb    {\mbox{\,nb$^{-1}$}}
\newcommand\Tesla  {\mbox{\,T}}
\newcommand\Teslam {\mbox{\,T${\CDOT}$m}}
\newcommand\Hz     {\ifmmode{\,\mathrm{Hz}}\else
                      \,\textrm{Hz}\fi}%
\newcommand\TeV    {\ifmmode{\,\mathrm{Te\kern -0.1em V}}\else
                      \,\textrm{Te\kern -0.1em V}\fi}%
\newcommand\GeV    {\ifmmode{\,\mathrm{Ge\kern -0.1em V}}\else
                      \,\textrm{Ge\kern -0.1em V}\fi}%
\newcommand\MeV    {\ifmmode{\,\mathrm{Me\kern -0.1em V}}\else
                      \,\textrm{Me\kern -0.1em V}\fi}%

% Stats
\newcommand\RMS          {r.m.s{.}}
\newcommand\CLs          {\mathrm{CL}_{\tiny S}}
\newcommand\likelihood   {\mathcal{L}}
\newcommand\Lmax         {\likelihood_\mathrm{max}}
\newcommand\BF           {\mathcal{B}}
\newcommand\Acc          {\mathcal{A}}
\newcommand\Cor          {\mathcal{C}}
\newcommand\Lint         {\mbox{\raisebox{2pt}{\footnotesize$\int$}} L\,{\rm{d}}t}
\newcommand\sigmu        {\upmu}
\newcommand\sigmuH       {\sigmu_{\scH}}
\newcommand\qmu          {q_\sigmu}

% Derivative is upright
\newcommand\D      {{\rm d}}

% Arrows
%   The original macro was written by Chris Quigg -> modified by tmhong.
%   This creates bent arrows to write sequential decays 
%   1 --> 2 3
%         | |--> 4
%         |--> 5
\newcommand\bentarrow{\mbox{$\:\raisebox{1.4ex}{\rlap{$\big\vert$}}\!\rightarrow\,$}}
\newcommand\bothdk[5]{
  \begin{array}{r c l}
    #1 & \!\rightarrow\! & #2 #3 \\
    & & \:\raisebox{1.4ex}{\rlap{$\big\vert$}}\raisebox{-0.5ex}{$\big\vert$}%
    \! \phantom{#2}\!\bentarrow #4 \\
    & & \:\raisebox{1.4ex}{\rlap{$\big\vert$}}\!\rightarrow #5
  \end{array}
}

% -----------------------------------------------------------------------------
% Variables

% Abbreviation
\newcommand\nova{NO$\nu$A~}

% MC generators
\newcommand\GEANT        {{\sc geant}}

% Known particle mass symbols

% Decay channels

% Regions
\newcommand\SR           {\textrm{\footnotesize\sc sr}}

% Super/subscripts

% Standard

% Special

% Fractions

% Efficiencies

% -----------------------------------------------------------------------------
% Colors

%\definecolor{myWW}{rgb}{0,0,255}

% -----------------------------------------------------------------------------
% Piechart

\newcommand{\slice}[6]{
  \pgfmathparse{0.5*#1+0.5*#2}
  \let\midangle\pgfmathresult

  % slice
  \ifthenelse{\equal{#2}{0}}{}{
    \draw[fill=#6] (0,0) -- (#1:1) arc (#1:#2:1) -- cycle;}

  % outer label
  \node[label=\midangle:#5] at (\midangle:0.95) {};

  % inner label
  \ifthenelse{#3 < 5}{}{
    \pgfmathparse{min((#2-#1-20)/110*(-0.3),0)}
    \let\temp\pgfmathresult
    \pgfmathparse{max(\temp,-0.4) + 0.8}
    \let\innerpos\pgfmathresult
      \node at (\midangle:\innerpos){
        \ifthenelse{\equal{blue}{#6}}{\color{white} #4}{
          \ifthenelse{\equal{magenta}{#6}}{\color{white} #4}{
            \ifthenelse{\equal{cyan}{#6}}{\color{white} #4}{
              \ifthenelse{\equal{black}{#6}}{\color{white} #4}{
                #4
              }
            }
          }
        }
      };
  }
}

% -----------------------------------------------------------------------------
% Documentation

% Clear two-column page
\makeatletter
\newcommand*{\balancecolsandclearpage}{%
  \close@column@grid
  \clearpage
  \twocolumngrid
}
\makeatother

% Repeating comments
\newcommand\HwwTableUpdate{\centering{\color{red}\large THIS TABLE WILL BE UPDATED}\\ }
\newcommand\HwwPlotUpdate{\centering{\color{red}\large FIGURE STYLE WILL BE UPDATED}\\ }
\newcommand\HwwPlotDetail[1]{#1 Fig.~\ref{fig:MET} for plotting details}
\newcommand\HwwPlotSelectionDetail[1]{#1 Figs.~\ref{fig:MET} and~\ref{fig:0j} for plotting details}
\newcommand\HwwPlotFitDetail[1]{#1 Figs.~\ref{fig:MET} and~\ref{fig:mT_fitDF} for details of plotting and normalizations}
\newcommand\HwwPlotFitDetailShort[1]{#1 Figs.~\ref{fig:MET} and~\ref{fig:mT_fitDF} for plotting details}
\newcommand\paper        {paper}
\newcommand\Paper        {Paper}

% Spacing
\newcommand\no           {\!\!}
\newcommand\np           {\no\no}
\newcommand\nq           {\np\np}
\newcommand\nqq          {\nq\nq\nq}
\newcommand\nqqq         {\nq\nq\nq\nq}
\newcommand\nr           {\phantom{i}\nq}
\newcommand\zz           {\phantom{..}}
\newcommand\z            {\phantom{0}}
\newcommand\Z            {\phantom{.0}}
\newcommand\QUAD         {~\,}

% Journals
\newcommand\Note[2]      {Report No.\ #2, \href{#1}{#1}}
\newcommand\CPC[1]       {Comput.~Phys.~Commun.~{\bf #1}}
\newcommand\EPJC[1]      {Eur.~Phys.~J.~C {\bf #1}}
\newcommand\JHEP[1]      {J.~High Energy Phys.~#1}
\newcommand\JINST[1]     {JINST {\bf #1}}
\newcommand\JPHYSG[1]    {J.~Phys.~{\bf G~#1}}
\newcommand\NIMA         {Nucl.~Instrum.~Methods~Phys.~Res.,~Sect.~A~}
\newcommand\NPhys[1]     {Nucl.~Phys.\ {\bf #1}}
\newcommand\PLB[1]       {Phys.~Lett.~B {\bf #1}}
\newcommand\PLett[1]     {Phys.~Lett.\ {\bf #1}}
\newcommand\PRep[1]      {Phys.~Rep.\ {\bf #1}}
\newcommand\PRev[1]      {Phys.~Rev.\ {\bf #1}}
\newcommand\PRD[1]       {Phys.~Rev.~D {\bf #1}}
\newcommand\PRL[1]       {Phys.~Rev.~Lett.\ {\bf #1}}
\newcommand\ZPC[1]       {Z.~Phys.~C {\bf #1}}
\newcommand\arxiv[1]     {\href{http://arxiv.org/abs/#1}{arXiv:#1}}
\newcommand\hepex[1]     {\href{http://arxiv.org/abs/hep-ex/#1}{hep-ex/#1}}
\newcommand\hepph[1]     {\href{http://arxiv.org/abs/hep-ph/#1}{hep-ph/#1}}

% Latin
\newcommand\ie           {i.\,e{.}}
\newcommand\eg           {e.\,g{.}}
\newcommand\etal         {{\it et al{.}}}
\newcommand\ibid         {{\it ibid{.}}}
\newcommand\insitu       {{\it in situ}}
\newcommand\cf           {cf{.}}

% Horizontal lines (double line, single line)
\newcommand\dbline{\noalign{\vskip 0.10truecm\hrule\vskip 0.05truecm\hrule\vskip 0.10truecm}}
\newcommand\sgline{\noalign{\vskip 0.10truecm\hrule\vskip 0.10truecm}}
\newcommand\clineskip{\noalign{\vskip 0.10truecm}}