% the abstract

\nova~is the current United States flagship long-baseline neutrino experiment designed to study the properties of neutrino oscillations. It consists of two functionally identical detectors each located 14.6 mrad off the central axis from the Fermilab NuMI neutrino beam. The Near Detector is located 1 km downstream from the beam source, and the Far Detector is located 810 km away in Ash River, Minnesota. This long baseline, combined with the ability of the NuMI facility to switch between nearly pure neutrino and anti-neutrino beams, allows \nova~to make precision measurements of neutrino mixing angles, potentially determine the neutrino mass hierarchy, and begin searching for CP violating effects in the lepton sector. However, \nova~can also probe more exotic scenarios, such as oscillations between the known active neutrinos and new sterile species.

This thesis showcases the first search for sterile neutrinos in a $3 + 1$ model at \nova. The analysis presented searches for a deficit in the rate of neutral current events at the Far Detector using the Near Detector to constrain the predicted spectrum. The comparison between the observed and predicted spectra is translated into a measurement of the expanded PMNS mixing matrix elements, $\Usqxy{\mu}{4}$ and $\Usqxy{\tau}{4}$, assuming a value of $\dmxy{1}{4} \sim O(1 \evsq)$. This analysis was performed using data taken between February 2014 and May 2015 corresponding to $3.52 \times 10^{20}$ protons on target. The best fit values for the matrix elements were $\Usqxy{\mu}{4} = 0.xy \pm a.bc$ and $\Usqxy{\tau}{4} = 0.vw \pm d.ef$, consistent with the no sterile neutrino hypothesis. At the end of this thesis there is a short discussion of future sensitivity improvements using a larger dataset.