% the abstract

\nova is a long baseline neutrino experiment designed to study neutrino oscillations. It consists of two functionally identical detectors each located 14 mrad off-axis from the NuMI neutrino beam generated at Fermilab, with one detector located about a kilometer from the beam source, and the other 810 km away in Ash River, Minnesota. With the longest distance between detectors and the ability of the NuMI beam to produce a beam of either neutrinos or anti-neutrinos, \nova is the most sensitive experiment to CP violating effects in the neutrino sector in the world. While the primary physics goals of \nova are to make measurements of the remaining unknown 3 flavor oscillation parameters, the experiment has the capability to perform more exotic analyses.

This thesis focuses on a search for sterile neutrinos in a $3 + 1$ model. The analysis presented searches for a deficit in the rate of neutral current events at the far detector using the near detector to constrain the predicted spectrum. The comparison between the observed and predicted spectra is translated into a measurement of the expanded PMNS mixing matrix elements, $\Usqxy{\mu}{4}$ and $\Usqxy{\tau}{4}$, assuming a value of $\dmxy{1}{4} \sim O(1 \evsq)$. This analysis was performed using data taken between February 2014 and May 2015 corresponding to $3.52 \times 10^{20}$ protons on target. The best fit values for the matrix elements were $\Usqxy{\mu}{4} = 0.xy \pm a.bc$ and $\Usqxy{\tau}{4} = 0.vw \pm d.ef$, consistent with the no sterile neutrino hypothesis. At the end of this thesis there is a short discussion of future sensitivity improvements using a larger dataset.