\chapter{Theory of Neutrino Oscillations}
\label{ch:Theory}

The idea of neutrino oscillations was first proposed by Pontecorvo in 1957 \cite{ref:Pontecorvo1}, but his proposal described oscillations between neutrinos and anti-neutrinos. In 1962, after the discovery of the muon neutrino, Maki, Nakagawa, and Sakata proposed the theory that described oscillations between neutrino flavors due to differing neutrino flavor and mass eigenstates \cite{ref:MNS}. This chapter describes the modern formalism in detail and uses natural units where $\hbar = c = 1$, except where otherwise noted.

\section{The PMNS Matrix}
\label{sec:TheoryPMNS}

In the Standard Model, neutrinos only interact via the W and Z bosons as shown by the Feynman diagrams in Fig.~\ref{fig:WZ}. From these diagrams, it is clear that neutrinos always interact in a definite flavor eigenstate, $\ket{\nu_\alpha}$. Furthermore, when a neutrino is produced from a W boson, the flavor is always determined by the associated charged lepton shown in eq.~\ref{eq:NuLepPairs}.

\beqa
\left( \begin{array}{c} \nue \\ e \end{array} \right), \quad \left( \begin{array}{c} \numu \\ \mu \end{array} \right), \quad \left( \begin{array}{c} \nutau \\ \tau \end{array} \right)
\label{eq:NuLepPairs}
\eeqa

\n On the other hand, neutrinos propagate through spacetime with a definite mass, $\ket{\nu_i}$ an eigenstate of the free Hamiltonian. The flavor states can be written as a superposition of the mass states via

\beq
\ket{\nu_\alpha} = \sum_{i=1}^n \Ustxy{\alpha}{i} \ket{\nu_i},
\label{eq:massflav}
\eeq

\n where $n$ is the number of neutrinos, and $U$ is the unitary PMNS matrix, named after Pontecorvo, Maki, Nakagawa, and Sakata. The PMNS matrix is unitary, and would reduce to the identity matrix if neutrinos did not oscillate between flavor states. Yet since it does provide the mechanism for flavor transitions, it can be thought of as analogous to the quark sector CKM matrix.

\section{Vacuum Oscillations}

Consider a neutrino in a state of definite flavor $\alpha$ at time $t = 0$, $\ket{\nu(0)} = \ket{\nu_\alpha}$. This state is in a superposition of mass eigenstates. The time evolution of this neutrino is simply the time evolution of the individual mass states. In a vacuum, this adds a phase factor to each mass state.

\beq
\ket{\nu_\alpha(t)} = \sum_{i} \Ustxy{\alpha}{i} e^{-i(E_i t - \mathbf{p_i \cdot x})} \ket{\nu_i}
\label{eq:NuAtT}
\eeq

\n With the neutrino at position $\mathbf{x} = L$ at time $t$, the dot product evaluates to $\mathbf{p_i \cdot x} = p_i L$. Eq~\ref{eq:NuAtT} can then simplified by making use of the fact that neutrinos are ultra-relativistic, allowing for several assumptions. First, the time, $t$, is replaced by the distance, $L$. Next, the energy of each mass state is approximated to be the same energy, $E_i = E$. Last, the momentum is expanded as $p_i = \sqrt{E^2 - m_i^2} \approx E - m_i^2/2E$. With these assumptions, eq.~\ref{eq:NuAtT} simplifies as:

\beq
\ket{\nu_\alpha(L)} = \sum_{i} \Ustxy{\alpha}{i} e^{-i m_i^2 L/2E} \ket{\nu_i}.
\label{eq:NuAtTRel}
\eeq

Eq.~\ref{eq:NuAtTRel} can be used to find the probability that the original neutrino in flavor state $\alpha$ has transitioned (or survived) as flavor state $\beta$. Using the conjugate of eq.~\ref{eq:massflav}, this state can be rewritten in terms of the mass eigenstates:

\beq
\bra{\nu_\beta} = \sum_{j = 1}^n \Uxy{\beta}{j} \bra{\nu_j}.
\label{eq:NuBra}
\eeq

\n Eq.~\ref{eq:NuBra} is then applied to eq.~\ref{eq:NuAtTRel}:

\beq
\braket{\nu_\beta | \nu_\alpha( L ) } = \sum_{i, j} \Ustxy{\alpha}{i} \Uxy{\beta}{j} e^{-i m_i^2 L/2E} \braket{\nu_j | \nu_i}
= \sum_i \Ustxy{\alpha}{i} \Uxy{\beta}{i} e^{-i m_i^2 L/2E}
\label{eq:nuAB}
\eeq

\n where the last equality follows from the orthogonality of individual mass eigenstates. The probability of the flavor transition is then the amplitude of this matrix element.

\beq
P(\nu_\alpha \rightarrow \nu_\beta) = \vert \braket{\nu_\beta | \nu_\alpha(L)} \vert^2
= \sum_{i, j} \Ustxy{\alpha}{i} \Uxy{\beta}{i} \Ustxy{\beta}{j} \Uxy{\alpha}{j} e^{-i (m_i^2 - m_j^2) L/2E}
\label{eq:nuABsq}
\eeq

Do some expansions.

\section{Standard 3-Flavor Oscillations}
\label{sec:Theory3}

\subsection{Matter Effects}
\label{sec:TheoryMatter}

\subsection{Current Measurements}
\label{sec:BestMeasures}

\section{Sterile Neutrinos}
\label{sec:Theory3+1}

\section{Neutrino Mass in the Standard Model}
