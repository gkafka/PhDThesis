\chapter{Theory of Neutrino Oscillations}
\label{ch:Theory}

The idea of neutrino oscillations was first proposed by Pontecorvo in 1957 \cite{ref:Pontecorvo1}, but his proposal described oscillations between neutrinos and anti-neutrinos. In 1962, after the discovery of the muon neutrino, Maki, Nakagawa, and Sakata proposed the theory that described oscillations between neutrino flavors due to differing neutrino flavor and mass eigenstates \cite{ref:MNS}. This chapter describes the modern formalism in detail and uses natural units where $\hbar = c = 1$, except where otherwise noted.

\section{The PMNS Matrix}
\label{sec:TheoryPMNS}

\begin{figure}[h]
  \begin{center} $
  \begin{array}{c c}
    {
    \begin{fmffile}{Wint}
      \begin{fmfgraph*}(150,150)
        \fmfleft{n1}
        \fmfright{l1}
        \fmfbottom{w1}
        \fmf{fermion}{n1,v1}
        \fmf{fermion}{v1,l1}
        \fmf{photon, tension=.5, label=$W^+$}{v1,w1}
        \fmflabel{$\nu_\ell$}{n1}
        \fmflabel{$\ell^-$}{l1}
      \end{fmfgraph*}
    \end{fmffile}
    }
    \quad \quad & \quad \quad
    {
    \begin{fmffile}{Zint}
      \begin{fmfgraph*}(150,150)
        \fmfleft{n1}
        \fmfright{l1}
        \fmfbottom{w1}
        \fmf{fermion}{n1,v1}
        \fmf{fermion}{v1,l1}
        \fmf{photon, tension=.5, label=$Z^0$}{v1,w1}
        \fmflabel{$\nu_\ell$}{n1}
        \fmflabel{$\nu_\ell$}{l1}
      \end{fmfgraph*}
    \end{fmffile}
    }
  \end{array} $
  \vspace{3 mm}
  \caption[Standard Model Neutrino Interaction Diagrams]{Standard Model Weak interactions involving a neutrino. Left: Charged current interaction. Right: Neutral current interaction.}
  \label{fig:WZ}
  \end{center}
\end{figure}

In the Standard Model, neutrinos only interact via the W and Z bosons as shown by the Feynman diagrams in Fig.~\ref{fig:WZ}. From these diagrams, it is clear that neutrinos always interact in a definite flavor eigenstate, $\ket{\nu_\alpha}$. Furthermore, when a neutrino is produced from a W boson, the flavor is always determined by the associated charged lepton shown in eq.~\ref{eq:NuLepPairs}.
\beq
\begin{pmatrix} \nue \\ e \end{pmatrix}, \quad \begin{pmatrix} \numu \\ \mu \end{pmatrix}, \quad \begin{pmatrix} \nutau \\ \tau \end{pmatrix}
\label{eq:NuLepPairs}
\eeq

\n On the other hand, neutrinos propagate through spacetime with a definite mass, $\ket{\nu_i}$ an eigenstate of the free Hamiltonian. The flavor states can be written as a superposition of the mass states via
\beq
\ket{\nu_\alpha} = \sum_{i=1}^n \Ustxy{\alpha}{i} \ket{\nu_i},
\label{eq:massflav}
\eeq

\n where $n$ is the number of neutrinos, and $U$ is the unitary PMNS matrix, named after Pontecorvo, Maki, Nakagawa, and Sakata. The PMNS matrix is unitary, and would reduce to the identity matrix if neutrinos did not oscillate between flavor states. Yet since it does provide the mechanism for flavor transitions, it can be thought of as analogous to the quark sector CKM matrix.

\section{Vacuum Oscillations}

In this section, the basics of neutrino oscillations are developed by considering oscillations in a vacuum. The neutrinos are treated as plane waves, as in \cite{ref:PlaneWaves}, with the assumption that the neutrino is actually localized in space put in by hand. A careful, rigorous analysis treating the neutrinos as plane waves in \cite{ref:WavePackets} reproduces the same results.

Consider a neutrino in a state of definite flavor $\alpha$ at time $t = 0$, $\ket{\nu(0)} = \ket{\nu_\alpha}$. This state is in a superposition of mass eigenstates. The time evolution of this neutrino is simply the time evolution of the individual mass states. In a vacuum, this adds a phase factor to each mass state.
\beq
\ket{\nu_\alpha(t)} = \sum_{i} \Ustxy{\alpha}{i} e^{-i(E_i t - \mathbf{p_i \cdot x})} \ket{\nu_i}
\label{eq:NuAtT}
\eeq

\n With the neutrino at position $\mathbf{x} = L$ at time $t$, the dot product evaluates to $\mathbf{p_i \cdot x} = p_i L$. Eq~\ref{eq:NuAtT} can then simplified by making use of the fact that neutrinos are ultra-relativistic, allowing for several assumptions. First, the time, $t$, is replaced by the distance, $L$. Next, the energy of each mass state is approximated to be the same energy, $E_i = E$. Last, the momentum is expanded as $p_i = \sqrt{E^2 - m_i^2} \approx E - m_i^2/2E$. With these assumptions, eq.~\ref{eq:NuAtT} simplifies as:
\beq
\ket{\nu_\alpha(L)} = \sum_{i} \Ustxy{\alpha}{i} e^{-i m_i^2 L/2E} \ket{\nu_i}.
\label{eq:NuAtTRel}
\eeq

\n The mass eigenstate inside the sum is then re-expressed in terms of flavor eigenstates using the inverse of eq.~\ref{eq:massflav} and unitarity of $U$.
\beq
\ket{\nu_\alpha(L)} = \sum_{\alpha'} \sum_{i} \Ustxy{\alpha}{i} \Uxy{\alpha'}{i} e^{-i m_i^2 L/2E} \ket{\nu_\alpha'}.
\label{eq:NuAtL}
\eeq

Eq.~\ref{eq:NuAtL} can then be used to find the probability that the original neutrino in flavor state $\alpha$ has transitioned (or survived) as flavor state $\beta$. First, the matrix element $\braket{\nu_\beta | \nu_\alpha (L)}$ is computed.
\beq
\braket{\nu_\beta | \nu_\alpha( L ) } = \sum_{\alpha'} \sum_{i} \Ustxy{\alpha}{i} \Uxy{\alpha'}{i} e^{-i m_i^2 L/2E} \braket{\nu_\beta | \nu_\alpha'}
= \sum_i \Ustxy{\alpha}{i} \Uxy{\beta}{i} e^{-i m_i^2 L/2E}
\label{eq:nuAB}
\eeq

\n The last equality in eq.~\ref{eq:nuAB} follows from the orthogonality of individual flavor eigenstates. The probability of the flavor transition is then the square of this matrix element.
\beq
P(\nu_\alpha \rightarrow \nu_\beta) = \vert \braket{\nu_\beta | \nu_\alpha(L)} \vert^2
= \sum_{i, j} \Ustxy{\alpha}{i} \Uxy{\beta}{i} \Ustxy{\beta}{j} \Uxy{\alpha}{j} e^{-i (m_i^2 - m_j^2) L/2E}
\label{eq:nuABsq}
\eeq

\n It is standard to rewrite the mass squared difference as $\dmxy{j}{i} \equiv m_i^2 - m_j^2$. Eq.~\ref{eq:nuABsq} is then manipulated using the properties of unitary matrices.
\beqa
P(\nu_\alpha \rightarrow \nu_\beta) & = & \sum_{i, j} \Ustxy{\alpha}{i} \Uxy{\beta}{i} \Ustxy{\beta}{j} \Uxy{\alpha}{j} + \sum_{i, j} \Ustxy{\alpha}{i} \Uxy{\beta}{i} \Ustxy{\beta}{j} \Uxy{\alpha}{j} (e^{-i \dmxy{j}{i} L/2E} - 1) \nonumber \\
& = & \delta_{\alpha\beta} + \sum_{i, j} \Ustxy{\alpha}{i} \Uxy{\beta}{i} \Ustxy{\beta}{j} \Uxy{\alpha}{j} (e^{-i \dmxy{j}{i} L/2E} - 1)
\label{eq:nuAB1minus}
\eeqa

\n The remaining summed term is further simplified making use of two facts. When $i = j$, the complex phase is 0 as $\dmxy{i}{i} = 0$, and thus these terms vanish. Second, the terms with $i < j$ are complex conjugates of those with $i > j$, and $z + z^* = 2\Re(z)$ for any complex number $z$.
\beqa
P(\nu_\alpha \rightarrow \nu_\beta) & = & \delta_{\alpha\beta} + 2\sum_{i > j} \Re \left[ \Ustxy{\alpha}{i} \Uxy{\beta}{i} \Ustxy{\beta}{j} \Uxy{\alpha}{j} (e^{-i \dmxy{j}{i} L/2E} - 1) \right]
\label{eq:nuABiGrj}
\eeqa

\n Both pieces of this term are split into their real and imaginary parts, and simplified using the trigonometric identity $\cos2\theta - 1 = -2\sin^2\theta$. Defining\hspace{0.5em}$\mathcal{U} \equiv \Ustxy{\alpha}{i} \Uxy{\beta}{i} \Ustxy{\beta}{j} \Uxy{\alpha}{j} (e^{-i \dmxy{j}{i} L/2E} - 1)$ and $\phi \equiv \dmxy{j}{i} L/2E$:
\beqa
\Re ( \mathcal{U} ) & = & \Re \left[ \Ustxy{\alpha}{i} \Uxy{\beta}{i} \Ustxy{\beta}{j} \Uxy{\alpha}{j} (e^{-i \dmxy{j}{i} L/2E} - 1) \right] \\
%& = & \Re \left\{ \left[ \Re ( \Ustxy{\alpha}{i} \Uxy{\beta}{i} \Ustxy{\beta}{j} \Uxy{\alpha}{j} ) + i \Im ( \Ustxy{\alpha}{i} \Uxy{\beta}{i} \Ustxy{\beta}{j} \Uxy{\alpha}{j} ) \right] \left[ \cos\phi - i\sin\phi - 1\right] \right\} \\
& = & \Re \left\{ \left[ \Re ( \Ustxy{\alpha}{i} \Uxy{\beta}{i} \Ustxy{\beta}{j} \Uxy{\alpha}{j} ) + i \Im ( \Ustxy{\alpha}{i} \Uxy{\beta}{i} \Ustxy{\beta}{j} \Uxy{\alpha}{j} ) \right] \left[ -2\sin^2 (\phi/2) - i\sin\phi \right] \right\} \\
%& = & \Re \left\{ -2 \Re ( \Ustxy{\alpha}{i} \Uxy{\beta}{i} \Ustxy{\beta}{j} \Uxy{\alpha}{j} ) \sin^2 (\phi/2) + \Im ( \Ustxy{\alpha}{i} \Uxy{\beta}{i} \Ustxy{\beta}{j} \Uxy{\alpha}{j} ) \sin\phi \right. \nonumber \\
%&& \quad\quad \left. -i \left[ \Re ( \Ustxy{\alpha}{i} \Uxy{\beta}{i} \Ustxy{\beta}{j} \Uxy{\alpha}{j} ) \sin\phi + 2\Im ( \Ustxy{\alpha}{i} \Uxy{\beta}{i} \Ustxy{\beta}{j} \Uxy{\alpha}{j} ) \sin^2 (\phi/2) \right] \right\} \\
& = & -2 \Re ( \Ustxy{\alpha}{i} \Uxy{\beta}{i} \Ustxy{\beta}{j} \Uxy{\alpha}{j} ) \sin^2 (\phi/2) + \Im ( \Ustxy{\alpha}{i} \Uxy{\beta}{i} \Ustxy{\beta}{j} \Uxy{\alpha}{j} ) \sin\phi
\label{eq:ExpandUUUUe1}
\eeqa

\n Inserting the expression from eq.~\ref{eq:ExpandUUUUe1} into eq.~\ref{eq:nuABiGrj}, we find:
\beqa
P(\nu_\alpha \rightarrow \nu_\beta) = \delta_{\alpha\beta} & - & 4\sum_{i > j} \Re ( \Ustxy{\alpha}{i} \Uxy{\beta}{i} \Ustxy{\beta}{j} \Uxy{\alpha}{j} ) \sin^2 \Delta_{ij} \nonumber \\
& + & 2\sum_{i > j} \Im ( \Ustxy{\alpha}{i} \Uxy{\beta}{i} \Ustxy{\beta}{j} \Uxy{\alpha}{j} ) \sin 2\Delta_{ij},
\label{eq:nuOsc}
\eeqa

\n where $\Delta_{ij} \equiv \Delta m^2_{ij} L / 4E = 1.267 \Delta m^2_{ij} L\mbox{ (km)} / E\mbox{ (GeV)}$. It can now be seen that the distance the neutrino travels, its energy, and the different mass splittings all affect the frequency of oscillation. Ideally, neutrino oscillations would be studied by having neutrinos with a fixed energy profile (preferably monoenergetic) and varying the baseline. However, neutrino detectors are incredibly large, so in practice the baseline is fixed and the oscillation probability is studied as a function of neutrino energy. 

For the case of survival probability, $\alpha = \beta$ and eq.~\ref{eq:nuOsc} simplifies further. The imaginary piece from eq.~\ref{eq:nuOsc} drops out, as
\beq
\Im ( \Ustxy{\alpha}{i} \Uxy{\alpha}{i} \Ustxy{\alpha}{j} \Uxy{\alpha}{j} ) = \Im ( \Usqxy{\alpha}{i} \Usqxy{\alpha}{j} ) = 0.
\label{eq:survIm}
\eeq

\n The survival probability is then given by:
\beq
P(\nu_\alpha \rightarrow \nu_\alpha) = 1 - 4 \sum_{i > j} \Usqxy{\alpha}{i} \Usqxy{\alpha}{j} \sin^2 \Delta_{ij}.
\label{eq:nuSurv}
\eeq

Due to the combined influence of mass splitting, oscillation baseline, and neutrino energy on the oscillation probability, it is often the case that only one term contributes to the sums in eq.s~\ref{eq:nuOsc} and \ref{eq:nuSurv}. The two neutrino approximation can be instructive in this instance. For this model, the mixing matrix simplifies to the two dimensional rotation matrix:
\beq
U = \begin{pmatrix} \cos\theta & \sin\theta \\ -\sin\theta & \cos\theta \end{pmatrix}.
\label{eq:2NuU}
\eeq

\n As this matrix is entirely real, the imaginary piece of eq.~\ref{eq:nuOsc} drops out. Plugging the matrix elements into the remaining term directly and simplifying slightly, we find the following forms for the survival and appearance probabilities.
\beqa
P(\nu_\alpha \rightarrow \nu_\alpha) & = & 1 - \sin^2 2\theta \sin^2 \left( \frac{\Delta m^2 L}{4E} \right) \label{eq:2NuSurv} \\
P(\nu_\alpha \nrightarrow \nu_\alpha) & = & \sin^2 2\theta \sin^2 \left( \frac{\Delta m^2 L}{4E} \right) \label{eq:2NuApp}
\eeqa

\n From these equations it is clear that the mixing matrix parameters control the amplitude of neutrino oscillations. For small angles, most neutrinos will not change flavor, while larger angles can cause most of the neutrinos to change flavor. The case where $\theta = 45^\circ$ is called maximal mixing as at specific baseline lengths the probability of oscillation becomes 1.

\section{Standard 3-Flavor Oscillations}
\label{sec:Theory3}

The Standard Model includes three neutrinos, so the PMNS matrix is $3 \times 3$ in this picture. Explicitly expanding eq.~\ref{eq:massflav}, $U$ takes the following form:
\beq
\begin{pmatrix} \nue \\ \numu \\ \nutau \end{pmatrix} = \begin{pmatrix} \Uxy{e}{1} & \Uxy{e}{2} & \Uxy{e}{3} \\ \Uxy{\mu}{1} & \Uxy{\mu}{2} & \Uxy{\mu}{3} \\ \Uxy{\tau}{1} & \Uxy{\tau}{2} & \Uxy{\tau}{3} \end{pmatrix} \begin{pmatrix} \nu_1 \\ \nu_2 \\ \nu_3 \end{pmatrix}.
\label{eq:FlavUMass}
\eeq

\n The PMNS matrix can be parametrized in terms of 3 real mixing angles, $\theta_{ij}$ and a complex phase, $\delta$, called the CP phase. Following the convention from the Particle Data Group \cite{ref:PDG}, the expanded matrix takes the form
\beqa
U & = & \begin{bmatrix} c_{12} c_{13} & s_{12} c_{13} & s_{13} e^{-i\delta} \\ -s_{12} c_{23} - c_{12} s_{23} s_{13} e^{i\delta} & c_{12} c_{23} - s_{12} s_{23} s_{13} e^{i\delta} & s_{23} c_{13} \\ s_{12} s_{23} - c_{12} c_{23} s_{13} s^{i\delta} & -c_{12} s_{23} - s_{12} c_{23} s_{13} e^{i\delta} & c_{23} c_{13} \end{bmatrix} \nonumber \\
& = & \begin{bmatrix} 1 & 0 & 0 \\ 0 & c_{23} & s_{23} \\ 0 & -s_{23} & c_{23} \end{bmatrix} \begin{bmatrix} c_{13} & 0 & s_{13} e^{-i\delta} \\ 0 & 1 & 0 \\ -s_{13} e^{i\delta} & 0 & c_{13} \end{bmatrix} \begin{bmatrix} c_{12} & s_{12} & 0 \\ -s_{12} & c_{12} & 0 \\ 0 & 0 & 1 \end{bmatrix}
\label{eq:3NuU}
\eeqa

\n where $c_{ij} \equiv \cos\theta_{ij}$ and $s_{ij} \equiv \sin\theta_{ij}$.

With three neutrinos, the expanded forms of eq.s \ref{eq:nuOsc} and \ref{eq:nuSurv} can still balloon into unwieldy messes. Fortunately, based on current knowledge of the mass splittings, it is usually the case that only one mass splitting scale matters and other terms can be dropped. Fig.~\ref{fig:MassSplit} shows a schematic of the mass splittings. For historic reasons, $\dmxy{1}{2}$ is known as the solar mass splitting and the larger mass splitting is called the atmospheric mass splitting. The atmospheric mass splitting is about 30 times the solar mass splitting. The sign of the solar mass splitting is known, while that of the atmospheric mass splitting is not. A positive value of $\dmxy{2}{3}$ is called the normal hierarchy; a negative value is called the inverted hierarchy.

\begin{figure}[h]
  \includegraphics[width=\textwidth]{figures/MassSplitting.png}
  \caption[Neutrino Mass Splitting Schematic]{A schematic of the mass splittings between the three known neutrino mass states and how much they couple to each of the flavor states \cite{ref:MassSplitRef}.}
  \label{fig:MassSplit}
\end{figure}

Two oscillation probabilities that are of interest to \nova~are the muon neutrino survival probability and electron neutrino appearance from a muon neutrino beam. Since $\vert\dmxy{1}{2}\vert$ is so much smaller than $\vert\dmxy{2}{3}\vert$, the solar oscillation baseline is much longer, thus the oscillation probability is first dominated by terms containing $\dmxy{2}{3}$. This is the case for \nova. Furthermore, the probability can be simplified by making making the assumption that $\vert \dmxy{2}{3} \vert \approx \vert \dmxy{1}{3} \vert$. Under these conditions, the survival probability of muon neutrinos is calculated as follows:
\beqa
P(\numu \rightarrow \numu) & \approx & 1 - 4\Usqxy{\mu}{3} (\Usqxy{\mu}{1} + \Usqxy{\mu}{2}) \sin^2 \Delta_{32} \label{eq:3MuToMu1} \\
& \approx & 1 - 4 s^2_{23} (1 - s^2_{13}) (c^2_{23} + s^2_{23} s^2_{13}) \sin^2 \Delta_{32} \label{eq:3MuToMu2} \\
& \approx & 1 - 4 s^2_{23} c^2_{23} \sin^2 \Delta_{32} + 4 s^2_{23} s^2_{13} (c^2_{23} - s^2_{23}) \sin^2 \Delta_{32} \label{eq:3MuToMu3} \\
& = & 1 - \sin^2 2\thxy{2}{3} \sin^2 \Delta_{32} + 4 \sin^2 \thxy{2}{3} \sin^2 \thxy{1}{3} \cos^2 2\thxy{2}{3} \sin^2 \Delta_{32} \quad\quad
\label{eq:3MuToMu}
\eeqa

\n Between eq.s \ref{eq:3MuToMu2} and \ref{eq:3MuToMu3}, the term proportional to $s^4_{13}$ was dropped using the current knowledge that $s^2_{13}$ is small \cite{ref:PDG}. Note that if $\theta_{13}$ were 0, then eq.~\ref{eq:3MuToMu} would reduce to eq.~\ref{eq:2NuSurv}, the two neutrino survival probability.

The full 3 flavor electron neutrino appearance from muon neutrino oscillation probability is often written in the form \cite{ref:Evan}:
\beq
P(\nuanu_{\mu} \rightarrow\,\nuanu_{e}) = P_{atm} + 2\sqrt{P_{atm}}\sqrt{P_{sol}} \left(\cos\delta \cos\Delta_{32}\, \varmp\, \sin\delta \sin\Delta_{32} \right) + P_{sol}
\label{eq:3MuToE}
\eeq

\n where
\beqa
\sqrt{P_{atm}} & \equiv & \sin \thxy{2}{3} \sin 2\thxy{1}{3} \sin \Delta_{32} \label{eq:Patm} \\
\sqrt{P_{sol}} & \equiv & \cos \thxy{2}{3} \sin 2\thxy{1}{2} \sin \Delta_{21} \label{eq:Psol}
\eeqa

\n where the approximation $\vert \dmxy{2}{3} \vert \approx \vert \dmxy{1}{3} \vert$ has been made and higher order terms of $s^2_{13}$ been dropped. For an experiment at a short enough baseline such as \nova, the $P_{sol}$ term is negligible as it depends on a higher order term of the solar mass splitting. The cross term is also not the dominant effect as it also depends upon the solar mass splitting, but it demonstrates interesting behavior. The $\cos\delta$ term is $CP$ conserving, but the $\sin\delta$ term exhibits $CP$ violation. This is why $\delta$ is called the $CP$ violating phase angle.

\section{Matter Effects}
\label{sec:TheoryMatter}

So far, the oscillation formalism has been developed only considering neutrinos in a vacuum. However, most neutrino oscillation experiments involve neutrinos traveling through matter, be it the Sun or the Earth. This affects the oscillation probabilities in a process called the Mikheyev-Smirnov-Wolfenstein effect, or MSW effect. The phenomenon was first proposed by Wolfenstein in 1978 \cite{ref:Wolfenstein}; Mikheyev and Smirnov built upon that work in 1985 \cite{ref:MSW} as a possible solution for the solar neutrino problem.

\begin{figure}[t]
  \begin{center} $
  \begin{array}{c c}
    {
    \begin{fmffile}{MSWnue}
      \begin{fmfgraph*}(150,120)
        \fmfleft{n1,l2}
        \fmfright{l1,n2}
        \fmf{fermion}{n1,v1,l1}
        \fmf{fermion}{l2,v2,n2}
        \fmf{photon, tension=.75, label=$W^-$}{v1,v2}
        \fmflabel{$\nue$}{n1}
        \fmflabel{$e$}{l1}
        \fmflabel{$\nue$}{n2}
        \fmflabel{$e$}{l2}
      \end{fmfgraph*}
    \end{fmffile}
    }
    \quad \quad & \quad \quad
    {
    \begin{fmffile}{MSWanue}
      \begin{fmfgraph*}(150,120)
        \fmfleft{n1,l1}
        \fmfright{n2,l2}
        \fmf{fermion}{l1,v1,n1}
        \fmf{fermion}{n2,v2,l2}
        \fmf{photon, tension=.75, label=$W^-$}{v1,v2}
        \fmflabel{$\nue$}{n1}
        \fmflabel{$e$}{l1}
        \fmflabel{$\nue$}{n2}
        \fmflabel{$e$}{l2}
      \end{fmfgraph*}
    \end{fmffile}
    }
  \end{array} $
  \vspace{3 mm}
  \caption[MSW Effect Interactions]{Coherent forward scattering interactions involved in the MSW effect. Left: Scattering of electron neutrinos on electrons. Right: Scattering of anti-electron neutrinos on electrons.}
  \label{fig:MSW}
  \end{center}
\end{figure}

The MSW effect is the coherent forward scattering of neutrinos off of the electrons in ordinary matter, a channel only available to electron flavor neutrinos and anti-neutrinos. Fig.~\ref{fig:MSW} illustrates the interactions. The electrons contribute an additional potential term, $V_e = \pm \sqrt{2}G_F N_e$, where $G_F$ is Fermi's constant, $N_e$ is the electron number density, the positive sign is for neutrinos, and the negative for anti-neutrinos. Neutrinos also forward scatter off the neutrons and protons in matter via neutral current interactions, but this only provides an overall phase as all neutrino flavors participate in these interactions equally. The matter induced potential adds an additional term to the Schr\"{o}dinger equation, affecting the time evolution of the flavor states and thus changing the oscillation probabilities.

The following derivation will consider the MSW effect in the case of two neutrino flavors. The time evolution of the flavor states is written as follows:
\beq
i \begin{pmatrix} \nue \\ \numu \end{pmatrix} = \left[ U \begin{pmatrix} \frac{m^2_1}{2E} & 0 \\ 0 & \frac{m^2_2}{2E} \end{pmatrix} U^\dagger + \begin{pmatrix} \pm V_e & 0 \\ 0 & 0 \end{pmatrix} \right] \begin{pmatrix} \nue \\ \numu \end{pmatrix}
\label{eq:2NuSchro}
\eeq

\n Inserting the 2 flavor PMNS matrix from eq.~\ref{eq:2NuU}, applying some trigonometric identites, and dropping common diagonal terms, eq.~\ref{eq:2NuSchro} simplifies to
\beq
i \begin{pmatrix} \nue \\ \numu \end{pmatrix} = \frac{1}{4E} \begin{pmatrix} -\dmxy{1}{2} \cos 2\theta \pm 4E V_e & \dmxy{1}{2} \sin 2\theta \\ \dmxy{1}{2} \sin 2\theta & \dmxy{1}{2} \cos 2\theta \end{pmatrix} \begin{pmatrix} \nue \\ \numu \end{pmatrix}.
\label{eq:2NuSchroNoDiag}
\eeq

\n The diagonal terms are dropped because they can be absorbed by the phase convention of the neutrino states. This Hamiltonian can be re-diagonalized with another unitary transformation, $H_{M} = U_{M}^\dagger H U_{M}$, with the following results:
\beqa
H_{M} & = & \frac{1}{2} \begin{pmatrix} -\frac{\Delta m^2_{M}}{2E} & 0 \\ 0 & \frac{\Delta m^2_{M}}{2E} \end{pmatrix} \label{eq:HMSW} \\
U_{M} & = & \begin{pmatrix} \cos \theta_{M} & \sin \theta_{M} \\ -\sin \theta_{M} & \cos\theta_{M} \end{pmatrix}, \label{eq:UMSW}
\eeqa

\n where
\beqa
\sin 2\theta_{M} & \equiv & \frac{\sin 2\theta}{A_{M}} \label{eq:thetaMSW} \\
\Delta m^2_{M} & \equiv & \dmxy{1}{2} A_{M} \label{eq:msqMSW} \\
A_{M} & \equiv & \sqrt{ \left( \cos 2\theta \mp \frac{2EV_e}{\dmxy{1}{2}} \right)^2 + \sin^2 2\theta } \label{eq:AMSW},
\eeqa

\n and now the negative sign in $A_M$ is for neutrinos and the positive sign for anti-neutrinos. As the electron number density goes to 0, so too does $V_e$ and the vacuum solution is recovered.

From the form of this solution, it can be seen that the Hamiltonian takes the same form as that in vacuum oscillations, but with modified effective masses. Likewise, $U_M$ has the same form as the 2 neutrino PMNS matrix, so $\theta_M$ can be considered the effective mixing angle. In the absence of neutrino oscillations (when $\theta = 0$), matter effects cannot ``create'' them. However, even for small angles $\theta$, the matter effect can create a resonant effect pushing the effective mixing angle, $\theta_M$, maximally to $45^\circ$. This occurs when the term in parenthesis in the definition of $A_M$ is $0$ (eq.~\ref{eq:AMSW}).
\beq
N_e^{res} = \frac{\dmxy{1}{2} \cos 2\theta}{2\sqrt{2} G_F E}
\label{eq:MSWres}
\eeq

In the case of 3 neutrinos, the same procedure is followed to diagonalize the Hamilton and obtain effective values for the various oscillation parameters. The effects are considerably more complicated, but the general effect is the same--matter changes the effective neutrino mass and alters the oscillation probability curves differently for neutrinos and anti-neutrinos. Under the same conditions that were used to calculate $P(\nuanu_\mu \rightarrow \nuanu_e)$ in sec.~\ref{sec:Theory3}, the results can be simplified to a few basic replacements \cite{ref:3NuMatter}.
\beq
P(\nuanu_{\mu} \rightarrow\,\nuanu_{e}) = P^M_{atm} + 2\sqrt{P^M_{atm}}\sqrt{P^M_{sol}} \left(\cos\delta \cos\Delta_{32}\, \varmp\, \sin\delta \sin\Delta_{32} \right) + P^M_{sol}
\label{eq:3MuToEMSW}
\eeq

\n This is exactly the same form as eq.~\ref{eq:3MuToE}. The interesting effects are seen with how $P^M_{atm}$ and $P^M_{sol}$ differ from their respective vacuum counterparts.
\beqa
\sqrt{P^M_{atm}} & \equiv & \sin \thxy{2}{3} \sin 2\thxy{1}{3} \frac{\sin (\Delta_{31} - aL)}{\Delta_{31} - aL} \Delta_{31} \label{eq:PatmMSW} \\
\sqrt{P_{sol}} & \equiv & \cos \thxy{2}{3} \sin 2\thxy{1}{2} \frac{\sin(aL)}{aL} \Delta_{21} \label{eq:PsolMSW}
\eeqa

\n Here, $a \equiv \pm G_F N_e / \sqrt{2}$ where the positive sign is for neutrinos and the negative sign for anti-neutrinos. For the Earth, $\vert a \vert \approx 1/3500\, km$.

The combined effect that appears in eq.s \ref{eq:3MuToEMSW}, \ref{eq:PatmMSW}, and \ref{eq:PsolMSW} due to the presence of matter plays an interesting role in the search for CP violation. The MSW effect by itself mimics CP violation as it alters oscillation probabilities for neutrinos and anti-neutrinos differently. Depending on the value of $\delta$ that nature has chosen, the differences in oscillation probabilities due to the CP violation angle and the MSW effect can either compound or cancel out.

\section{Current Measurements}
\label{sec:BestMeasures}

Most of the free parameters in the PMNS matrix have been measured by various solar, atmospheric, accelerator, and reactor neutrino experiments. However, any given neutrino experiment does not have sensitivity to all of the oscillation parameters. Instead, experiments are sensitive to specific angles based on their baseline and the energies of the neutrinos they observe. Solar neutrino experiments, such as GALLEX, SAGE, Super-K, and SNO, measure neutrinos with energies on the order of several MeV after a very long baseline, and are most sensitive to $\theta_{12}$ and $\dmxy{1}{2}$. Due to the strong MSW effect within the Sun, solar neutrino experiments also determined the ordering of mass states $\nu_1$ and $\nu_2$; $\nu_2$ is defined as the heavier state. Atmospheric neutrino experiments, such as Super-K, SNO, and MINOS, measure neutrinos generated by cosmic ray collisions with the Earth's atmosphere, and are sensitive to $\theta_{23}$ and $\dmxy{2}{3}$.

Reactor neutrino experiments, such as Chooz, Double Chooz, RENO, and Daya Bay, measure $\anue$ generated by nearby nuclear reactors. Like solar neutrinos, reactor neutrinos have energies on the order of a few MeV. By measuring these neutrinos with a short baseline ($O(1 km)$), the 2 neutrino approximation is valid, so the oscillation probability can be approximated as:
\beq
P(\anue \rightarrow \anue) \approx 1 - \sin^2 2\theta_{13} \sin^2 \Delta_{31}.
\label{eq:2NuReactor}
\eeq

\n Daya Bay made the first nonzero measurement of $\theta_{13}$ in 2012, reporting a value of $\sin^2 2\theta_{13} = 0.092 \pm 0.016\mbox{ (stat)} \pm 0.005\mbox{ (syst)}$ after taking just 55 days of data \cite{ref:DayaBay2012}. This result excluded a zero value for $\theta_{13}$ at $5.2\sigma$ and is shown in Fig.~\ref{fig:DayaBay2012}. Since that result, the limits have only continued to improve, and the leading measurement still comes from Daya Bay.

\begin{figure}[h]
  \centering
  \includegraphics[width=0.75\textwidth]{figures/DayaBay2012.png}
  \caption[First Measurement of $\theta_{13}$ from Daya Bay]{First measurement of $\theta_{13}$ from Daya Bay \cite{ref:DayaBay2012}. The points show the ratio of observed to expected events assuming $\theta_{13} = 0$. Each point is a measurement at a different detector. The inset in the upper corner shows the $\chi^2$ value vs value of $\sin^2 2\theta_{13}$, and excludes $\theta_{13} = 0$ at greater than $5\sigma$.}
  \label{fig:DayaBay2012}
\end{figure}

Accelerator neutrino experiments, such as MINOS, T2K, and \nova, begin with a beam of nearly pure $\nuanu_\mu$ and search for both a disappearance of $\nuanu_\mu$ and appearance of other neutrino flavors. These experiments are sensitive to $\theta_{13}$, $\theta_{23}$, $\dmxy{2}{3}$, and $\delta$. The experiments that have the largest matter effect are the most sensitive to $\delta$ and the mass hierarchy.

Global fits to the combined data of these (and other) neutrino experiments have been performed and summarized in \cite{ref:PDG, ref:BestFits3}; the best fit values are shown in Table \ref{tab:BestFits3}. While most of the parameters have been measured with good precision, there are still a few lingering questions. From the table it is clear that a much better measurement on the CP violation angle is needed. The mass hierarchy still needs to be definitively measured as well. The other main question is whether $\theta_{23}$ is maximal, and if not, whether it is in the lower or upper octant.
\begin{table}[h]
  \begin{center}
    \caption[Best Fit Parameters for Three Neutrino Oscillation Model]{Current status of best fit oscillation parameters, from \cite{ref:PDG, ref:BestFits3}. The last column shows the allowed values within a $3\sigma$ range, with the exception of $\delta$, which is shown at a $2\sigma$ range. This is because the current global best fit for $\delta$ still allows the full range from $0$ to $2\pi$ at $3\sigma$. \nova~should vastly improve the limits on $\delta$.}
    \label{tab:BestFits3}
    \begin{tabular}{l c l l}
      \hline\hline
      \multicolumn{2}{c}{Parameter} & \multicolumn{1}{c}{Best-Fit $(\pm 1\sigma)$} & \multicolumn{1}{c}{$3\sigma$ Range} \\
      \hline
      $\dmxy{1}{2} \left[10^{-5}\evsq\right]$ && $7.54^{+0.26}_{-0.22}$ & $6.99 - 8.18$ \\
      \multirow{2}{*}{$\vert \Delta m^2 \vert \left[10^{-3}\evsq\right]$} & NH & $2.43 \pm 0.06$ & $2.23 - 2.61$ \\
      & IH & $2.38 \pm 0.06$ & $2.19 - 2.56$ \\
      $\sin^2 \theta_{12}$ && $0.308 \pm 0.017$ & $0.259 - 0.359$ \\
      \multirow{2}{*}{$\sin^2 \theta_{23}$} & NH & $0.437^{+0.033}_{-0.023}$ & $0.374 - 0.628$ \\
      & IH & $0.455^{+0.039}_{0.031}$ & $0.380 - 0.641$ \\
      \multirow{2}{*}{$\sin^2 \theta_{13}$} & NH & $0.0234^{+0.0020}_{-0.0019}$ & $0.0176 - 0.0295$ \\
      & IH & $0.0240^{+0.0019}_{-0.0022}$ & $0.0178 - 0.0298$ \\
      \multirow{2}{*}{$\delta/\pi$ ($2\sigma$ range)} & NH & $1.39^{+0.38}_{-0.27}$ & $(0.00 - 0.16) \oplus (0.86 - 2.00)$ \\
      & IH & $1.31^{+0.29}_{-0.33}$ & $(0.00 - 0.02) \oplus (0.70 - 2.00)$\\
      \hline
    \end{tabular}
  \end{center}
\end{table}

Current and next generation reactor experiments have a great outlook to answer these outstanding questions. Making use of the MSW effect, the $\nue$ and $\anue$ appearance results from experiments like T2K and \nova~could simultaneously measure $\delta$, the mass hierarchy, and $\theta_{23}$ octant. The prospects for \nova to make these measurements are shown in Fig.~\ref{fig:BiProb}. The first analysis results from \nova~\cite{ref:NOvAFANuE, ref:NOvAFANuMu} were published after taking about 10\% of the experiments design statistics and already show promise. The \nova~measurement for $\delta$, shown in Fig.~\ref{fig:NOvAFADelta}, provide a hint toward the normal hierarchy and eliminate portions of $\delta$ space at 90\% confidence.

\begin{figure}[h]
  \centering
  \begin{subfigure}{.48\textwidth}
    \centering
    \includegraphics[width=1\linewidth]{figures/BiProbability.png}
  \end{subfigure}
  \begin{subfigure}{.48\textwidth}
    \centering
    \includegraphics[width=1\linewidth]{figures/BiProbabilityVary23.png}
  \end{subfigure}
  \caption[Bi-Probability Plots for $\nue$ Appearance]{Probability of $\nue$ appearance versus $\anue$ appearance at \nova. The blue ellipses are for the normal hierarchy; the red ellipses are for the inverted hierarchy. The starred points show a possible measurement \nova~could make. The matter effect can either constructively or destructively combine with the CP violation effect. A larger matter effect, further separates the two mass hierarchy ellipses. This corresponds to neutrinos passing through more matter. On the left, $\theta_{23}$ is assumed to be $45^\circ$ for maximal mixing, purely showcasing the interference between the matter and CP violation effects. On the right, $\theta_{23}$ is non-maximal, showing how the dependence on $\theta_{23}$ affects both ellipses in the same way.}
  \label{fig:BiProb}
\end{figure}

\begin{figure}[h]
  \centering
  \includegraphics[width=0.75\textwidth]{figures/NOvAFADelta.png}
  \caption[First Measurement of $\delta$ by \nova]{First measurement of $\delta$ from \nova~\cite{ref:NOvAFANuE}. The plot shows the significance of the difference between the observed and predicted number of events as a function of delta. \nova~used a primary and secondary selection technique, the primary (secondary) technique is shown as the solid (dotted) line. The secondary selection disfavors the inverted hierarchy for all values of $\delta$.}
  \label{fig:NOvAFADelta}
\end{figure}

\section{Sterile Neutrinos}
\label{sec:Theory3+1}

\section{Neutrino Mass in the Standard Model}
