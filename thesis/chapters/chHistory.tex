%\begin{savequote}[75mm]
%Nulla facilisi. In vel sem. Morbi id urna in diam dignissim feugiat. Proin molestie tortor eu velit. Aliquam erat volutpat. Nullam ultrices, diam tempus vulputate egestas, eros pede varius leo.
%\qauthor{Quoteauthor Lastname}
%\end{savequote}

\chapter{A Brief History of Neutrinos}

%\newthought{There's something to be said} for having a good opening line. Morbi commodo, ipsum sed pharetra gravida, orci  $x = 1/\alpha$ magna rhoncus neque, id pulvinar odio lorem non turpis \cite{Eigen1971, Knuth1968}. Nullam sit amet enim. Suspendisse id velit vitae ligula volutpat condimentum. Aliquam erat volutpat. Sed quis velit. Nulla facilisi. Nulla libero. Vivamus pharetra posuere sapien. Nam consectetuer. Sed aliquam, nunc eget euismod ullamcorper, lectus nunc ullamcorper orci, fermentum bibendum enim nibh eget ipsum. Donec porttitor ligula eu dolor. Maecenas vitae nulla consequat libero cursus venenatis. Nam magna enim, accumsan eu, blandit sed, blandit a, eros.
%$$\zeta = \frac{1039}{\pi}$$

\section{Introduction}

The neutrino was first postulated by Wolfgang Pauli as a possible explanation for the continuous spectrum of electrons emitted from nuclear $\beta$ decay \cite{ref:Pauli}. This decay was originally thought to be the emission of an electron from an atom, resulting in a different nucleus, via the process,

\beq
N \rightarrow N\ ' + e
\label{eq:BetaWrong}
\eeq

\n where $N$ and $N\,'$ are the parent and daughter nuclei, respectively. In a two body decay such as this, the momenta and energies of the outgoing particles are exactly constrained. Pauli's new particle explained the continuous spectrum of electron energy via a modified decay process:

\beq
N \rightarrow N\ ' + e + \nu
\label{eq:BetaRight}
\eeq

\n where $\nu$ is the outgoing neutral particle. Pauli's original proposal called the new particle the neutron, but this name was later used to name the massive neutral nucleon discovered by Chadwick in 1932 \cite{ref:Chadwick}. Three years after Pauli's idea, Fermi proposed a model for nuclear $\beta$ decay that included the new particle, which he coined the neutrino, or little neutral one \cite{ref:Fermi}.

\section{First Detection of Neutrinos}

Twenty years passed from Fermi's model proposal before neutrinos were discovered experimentally. Reines and Cowan made the discovery by placing a detector near a nuclear reactor as a source of neutrinos and observing inverse $\beta$ decay \cite{ref:1953, ref:1956}. The neutrinos observed were anti-electron neutrinos, thus the following was the observed process.

\beq
p + \anue \rightarrow n + e^{+}.
\label{eq:BetaInv}
\eeq

\n Fred Reines earned the Nobel Prize in Physics in 1995 for the detection of the neutrino.

In 1962, the muon neutrino was discovered at Brookhaven National Laboratory using the first neutrino beam \cite{ref:BNL} in a scheme still used in neutrino experiments today. The beam was generated by colliding protons with a target, producing pions that decayed into muons and muon neutrinos. The resultant beam then passed through thick steel, absorbing everything but the neutrinos. Leon Lederman, Melvin Schwartz, and Jack Steinberger won the Nobel Prize in Physics in 1988 for the discovery of the muon neutrino.

The last generation of neutrino, the tau neutrino, was discovered at Fermilab by the DONUT collaboration in 2000 \cite{ref:DONUT}.

\section{First of Evidence of Oscillations}

\section{Possible Evidence of Sterile Neutrinos}

The number of active neutrinos is constrained by measurements of the width of the Z boson. LEP has measured the number of active neutrinos to be $2.984 \pm 0.008$ \cite{ref:LEP}, so the discoveries of the $\nue$, $\numu$, and $\nutau$ leave no room for new active neutrinos.

% For an example of a full page figure, see Fig.~\ref{fig:myFullPageFigure}.

%\texttt{This is a line of code.}

%\begin{figure}
%\includegraphics[width=\textwidth]{figures/fig1}
%\caption[Short figure name.]{This is a figure that floats inline and here is its caption.
%\label{fig:myInlineFigure}}
%\end{figure}

%% Requires fltpage2 package
%%
% \begin{FPfigure}
% \includegraphics[width=\textwidth]{figures/fullpage}
% \caption[Short figure name.]{This is a full page figure using the FPfigure command. It takes up the whole page and the caption appears on the preceding page. Its useful for large figures. Harvard's rules about full page figures are tricky, but you don't have to worry about it because we took care of it for you. For example, the full figure is supposed to have a title in the same style as the caption but without the actual caption. The caption is supposed to appear alone on the preceding page with no other text. You do't have to worry about any of that. We have modified the fltpage package to make it work. This is a lengthy caption and it clearly would not fit on the same page as the figure. Note that you should only use the FPfigure command in instances where the figure really is too large. If the figure is small enough to fit by the caption than it does not produce the desired effect. Good luck with your thesis. I have to keep writing this to make the caption really long. LaTex is a lot of fun. You will enjoy working with it. Good luck on your post doctoral life! I am looking forward to mine. \label{fig:myFullPageFigure}}
% \end{FPfigure}
% \afterpage{\clearpage}
